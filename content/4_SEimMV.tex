\chapter{Social Engineering im Metaverse}\label{ch:SocialEngineeringimMV}

\section{Das Metaverse als Ziel für Social Engineering}

\subsection{Was macht das Metaverse interessant für Social Engineering}

\subsubsection{Charakterisierung der möglichen Zielpersonen}

\subsection{Gefahren für Minderjährige im Metaverse}

\section{Anwendungsmöglichkeiten von Social Engineering im Metaverse}

\subsubsection{Identitätsdiebstahl}
TODO

Beispiel Warframe

Im August 2022 wurde ich Zeuge, wie Social Engineering das Vertrauen seiner Opfer gezielt ausnutzt. Ein junges Mädchen, das neu im Spiel war, wurde von zwei erfahrenen Clan-Mitgliedern durch kleine Gefälligkeiten unterstützt. Sie schenkten ihr virtuelle Artefakte, die für neue Spieler sehr nützlich, für erfahrene Spieler jedoch wertlos sind. Zudem erklärten sie ihr, wie die verschiedenen Missionen im Spiel funktionieren und worauf sie achten muss.

Über ein halbes Jahr hinweg spielte das Mädchen regelmäßig mit den beiden und sammelte dabei eine Reihe wertvoller Ingame-Items. Da einige Aufgaben im Spiel zu schwierig für sie waren, boten die beiden Männer an, sich kurzzeitig in ihren Xbox-Account einzuloggen, um die schwierigen Missionen für sie zu erledigen. Sie wiesen sie sogar darauf hin, ein temporäres Passwort einzurichten, das sie nach erfolgreicher Mission wieder ändern könne. Durch Reziprozität und Sympathie beeinflusst, vertraute das Mädchen den Männern so sehr, dass sie schließlich ihre tatsächlichen Accountdaten weitergab, überzeugt davon, dass sie ihnen vertrauen konnte.

In ihrer Abwesenheit plünderten die Männer heimlich kleine Mnegen ihrer Ingame-Items und betrieben in ihrem Namen Handel. Zunächst bemerkte das Mädchen nichts davon. Erst während eines Urlaubs, als die Männer es übertrieben und auffällige Aktivitäten auf ihrem Account stattfanden, flog der Betrug auf. Die Accounts der Männer wurden daraufhin gesperrt, das Mädchen verzichtete jedoch auf eine Anzeige.


\subsection{Deep Fakes}
TODO

Avatar ist ein Gegenstand und kann lauschen
\subsection{Manipulation durch Gamification-Elemente}
\subsection{Biometrische Hacks}
TODO

Brillen können gehackt werden Biometrische Daten ausgelesen werden wie mimiken etc 

\subsection{blockchain Hacks}

Von Bedeutung ist dabei, dass es keine Kontrollinstanz gibt, die eingreifen kann, wenn der private Schlüssel verloren geht oder entwendet wird. Zudem bieten Wallet-Adressen allein keine ausreichende Identifikation der beteiligten Parteien.

\section{Auswirkungen des Social Engineering im Metaverse}

\subsection{persönliche Auswirkungen}
\subsection{soziale Auswirkungen}
\subsection{wirtschaftliche Auswirkungen}

\section{Fallbeispiel}

Fallstudie: Der Angriff auf VirtuCon
Hintergrund
VirtuCon war eine großangelegte virtuelle Konferenz im Metaverse, die auf einer populären Plattform für digitale Zusammenkünfte und Veranstaltungen stattfand. Die Konferenz zog Tausende von Teilnehmern an, darunter führende Experten in den Bereichen Technologie, Wirtschaft und Bildung. VirtuCon bot eine Vielzahl von Sitzungen, Workshops und Networking-Möglichkeiten in einer vollständig immersiven 3D-Umgebung.
Der Angriff
Einige Tage vor der Veranstaltung begannen die Organisatoren, Berichte über gefälschte Veranstaltungseinladungen zu erhalten, die an Teilnehmer gesendet wurden. Diese Einladungen enthielten Links, die angeblich zu exklusiven Vorregistrierungsboni oder speziellen Zugängen für die Konferenz führten. Tatsächlich leiteten diese Links die Nutzer jedoch auf gefälschte Login-Seiten, die darauf abzielten, persönliche Daten und Zugangsinformationen zu stehlen.
Parallel dazu schafften es die Angreifer, während der Veranstaltung mehrere Avatare zu kapern. Diese gekaperten Avatare wurden genutzt, um in verschiedenen Sitzungen und Chaträumen anwesend zu sein, wo sie weiterhin gefälschte Links verbreiteten und sogar versuchten, in private Gespräche einzudringen, um vertrauliche Informationen zu erlangen.
Analyse
Die Angreifer nutzten eine Kombination aus Phishing-Techniken und der Übernahme von Avataren, um das Vertrauen der Teilnehmer zu gewinnen und sie zur Preisgabe sensibler Informationen zu verleiten. Die Immersion und das Engagement im Metaverse trugen dazu bei, dass die Teilnehmer weniger misstrauisch gegenüber den ungewöhnlichen Aktivitäten waren, da sie die Interaktionen als Teil der Konferenzerfahrung ansahen.
Die psychologischen Tricks, die dabei zum Einsatz kamen, umfassten das Vorspiegeln von Dringlichkeit (durch das Angebot von "exklusiven Boni"), die Nutzung von Autorität und Vertrautheit (durch das Kapern bekannter Avatare) und das Ausnutzen der Neugier und des Wunsches nach Vernetzung der Teilnehmer.
Lessons Learned und Ableitungen für die Zukunft
Diese Fallstudie unterstreicht die Notwendigkeit umfassender Sicherheitsmaßnahmen und Awareness-Programme für Teilnehmer und Organisatoren von Veranstaltungen im Metaverse. Dazu gehören:
	• Verifizierung und Authentifizierung: Die Implementierung robuster Verifizierungs- und Authentifizierungsverfahren für alle Teilnehmer, Inhalte und Interaktionen.
	• Aufklärung und Training: Die Sensibilisierung der Nutzer für die Risiken und Anzeichen von Social Engineering-Angriffen.
	• Technische Sicherheitslösungen: Die Nutzung von Sicherheitstechnologien, um den Zugriff auf Veranstaltungen zu sichern und die Kommunikation zwischen Teilnehmern zu schützen.
